The code defines several functions for image compression and decompression. 

The main function is \texttt{compress}, which takes as input an image file and a compressed file, and produces a list of tuples as output. 

The \texttt{compress} function works as follows:

\begin{enumerate}
    \item It opens the input image file and gets the pixel data with the \texttt{getdata} method.
    \item It iterates through the pixel values and counts the number of consecutive pixels that have the same value.
    \item It stores this information as a tuple with the count and the pixel value. If the pixel value changes, it stores the information for the previous pixel value and starts counting the new pixel value.
    \item It opens the compressed file and writes the width and height of the image, as well as some metadata, to the file.
    \item It writes the list of tuples to the compressed file.
\end{enumerate}

The \texttt{descomprimido} function takes as input a compressed file and produces a decompressed image file as output. It works as follows:

\begin{enumerate}
    \item It reads the compressed file and extracts the pixel count information from it.
    \item It writes the pixel values to an output file, alternating between 0s and 1s depending on the count information.
    \item It also writes some metadata to the output file, including the image width and height, which are obtained from the \texttt{get\_size} function.
\end{enumerate}

Overall, this code implements a simple form of image compression and decompression by replacing runs of identical pixels with a count of the number of pixels and the pixel value. It writes the compressed information to a text file, and reads and processes this file to produce the decompressed image.
